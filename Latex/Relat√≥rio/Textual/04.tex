\section{Metodologia}

Variáveis explanatórias (e suas versões log-transformadas) também serão empregadas. O número de habitantes, a área de cada município e as coordenadas geográficas de suas sedes serão obtidas no sítio do IBGE (2020). Os biomas predominantes em cada município serão determinados de acordo com o sítio MapBiomas (2020).

Análises univariadas descreverão a distribuição de cada descritor para os municípios valendo-se de estatísticas de posição (média, mediana) e dispersão (amplitude, interquartis, desvio-padrão), visualizadas por meio de histogramas \cite{NicholasJ.Gotelli;AaronM.Ellison2010,Borcard2011}. Análises bivariadas descreverão as relações das variáveis pareadas, por meio de correlação e regressão, em versões paramétricas ou não \cite{NicholasJ.Gotelli;AaronM.Ellison2010,Borcard2011}. Comparações entre curvas serão feitas por análise de covariância \cite{NicholasJ.Gotelli;AaronM.Ellison2010,Borcard2011}.

Análises multivariadas descreverão as relações de similaridade entre municípios quanto à constituição específica e quanto aos valores dos fatores explanatórios \cite{NicholasJ.Gotelli;AaronM.Ellison2010,Borcard2011}. Comparações entre matrizes de similaridade serão feitas por meio do teste de Mantel \cite{Borcard2011}. Mapas temáticos apresentarão as variáveis em sua distribuição geográfica. As abordagens discriminadas serão conduzidas com os programas R \cite{CoreTeam2017}.


\subsection{Materiais e Métodos}

No período de janeiro a fevereiro de 2020, foram coletados os registros provenientes de cientistas-cidadãos no portal Wikiaves (WikiAves, 2020) e aqueles provenientes de cientistas nas coleções disponíveis na rede SpeciesLink (2020). Para equiparação dos bancos de dados, as espécies foram denominadas em acordo com a lista das aves do Brasil do Comitê Brasileiro de Registros Ornitológicos \cite{De2015}.

Para responder às questões supracitadas, foram utilizadas as seguintes variáveis oriundas dos dois bancos de dados: o número de registros de cada espécie por município, o número de espécies registradas por município nas suas versões log-transformadas (base 10). Foram empregadas, também, variáveis explanatórias, sendo elas: A altitude da sede, logaritmo na base 10 área, logaritmo na base 10 da quantidade de habitantes, latitude e longitude referentes a cada município. Todas as informações quanto as variáveis explanatórias foram obtidas no sítio do IBGE (2020).

\subsection{Etapas da pesquisa}

\subsubsection{Análise Univariada}
