\section{Metodologia}

\subsection{Materiais e Métodos}

No período de janeiro a fevereiro de 2020, foram coletados os registros provenientes de cientistas-cidadãos no portal Wikiaves (WikiAves, 2020) e aqueles provenientes de cientistas nas coleções disponíveis na rede SpeciesLink (2020).

Para responder às questões supracitadas, foram utilizadas as seguintes variáveis oriundas dos dois bancos de dados: o número de registros de cada espécie por município, o número de espécies registradas por município e suas versões log-transformadas (base 10). Foram empregadas, também, variáveis explanatórias, sendo elas: a altitude da sede, logaritmo na base 10 área, logaritmo na base 10 da quantidade de habitantes, latitude e longitude referentes a cada município. Todas as informações quanto as variáveis explanatórias foram obtidas no sítio do IBGE (2020). Os biomas predominantes em cada município serão determinados de acordo com o sítio MapBiomas (2020).

 As abordagens discriminadas serão conduzidas com os programas R \cite{CoreTeam2017}.

\subsection{Etapas da pesquisa}

\color{blue}\textit{Aqui começam a surgir algumas dúvidas, não mantive a folha de considerações para não atrapalhar a estrutura do texto, vou colocar as considerações ao longo do texto, porém, em itálico e azul.\\ A primeira delas é que eu não sei se posso adicionar subsubseções ao modelo da ufabc, acredito que não, de todo modo o fiz neste primeiro momento por um questão de organização, apenas.}\color{black}

A principio, para equiparação dos bancos de dados, as espécies foram denominadas em acordo com a lista das aves do Brasil do Comitê Brasileiro de Registros Ornitológicos \cite{De2015}.

\subsubsection{Análise Univariada}

Para medir a normalidade dos dados coletados foi empregada uma análise exploratória de dados \cite{NicholasJ.Gotelli;AaronM.Ellison2010,Field2012, Borcard2011}. Foram calculados, portanto, as estatísticas de posição (média, mediana e quartis), dispersão (desvio padrão), inclinação e curtose.  \color{blue}\textit{(Ok, mais um grande parenteses aqui, eu não falei de inclinação e curtose até agora porque (me perdoe) eu tinha esquecido desses valores, eles estavam na planilha que eu estava usando quando fiz as análises no past, ela foi arquivada e eles ficaram perdidos. Para escrever o trabalho de maneira mais formal, tive que reler algumas referências e lembrei que esses valores podem ser importantes (o valor de curtose no WAV2 para a quantidade de espécies é de 11.50854, o que eu acho que é um valor alto), se eles forem importantes, posso apresenta-los nas tabelas.)}\color{black}. Foram removidos os valores discrepantes relativos às variáveis explanatórias \cite{VALENTIN2000,Field2012} e aplicada a transformação logarítmica para todas as variavéis exceto: altitude, latitude e longitude \cite{NicholasJ.Gotelli;AaronM.Ellison2010,Field2012,Borcard2011}. \color{blue}\textit{(Anteriormente eu falei que foram utilizadas as versões log-transformadas. Acredito que isto seja um pleonasmo, mas não sei se esta informação deve aparecer aqui ou lá.)}\color{black}. Os dados foram impressos como gráficos de linhas. \color{blue}\textit{(Eu gostei da apresentação dos gráficos de linha, mas, não me lembro de nenhum livro que justifique isto, então deixei sem citação.)}\color{black}



\subsubsection{Análise Bivariada}

A partir dos dados reformatados após a análise anterior, foi estabelecido um exame quanto a regressão linear dos parâmetros apresentados \cite{NicholasJ.Gotelli;AaronM.Ellison2010}. Foram medidos os valores de $r^2$, $p$, a inclinação da reta e o intercepto para como cada variável preditora (quantidade de registros e espécies próprias de cada banco de dados) modifica-se conforme alteração nas variáveis explanatórias e como as variáveis preditoras modificam-se entre si, isto é, variação da quantidade de registros de um banco de dados (analogamente, espécies) conforme alteração da quantidade de registros do outro. Os pares-ordenados com resíduos discrepantes foram excluídos \cite{Field2012}.

Por fins de encontrar a variação na quantidade de registros conforme a quantidade de espécies adotou-se um modelo de regressão não linear \cite{NicholasJ.Gotelli;AaronM.Ellison2010}, a partir deste, foram calculados os valores de $r^2$, $p$, inclinação da reta e intercepto.

Para uma análise visual dos dados foram aplicados gráficos de dispersão ao modelo de regressão \cite{NicholasJ.Gotelli;AaronM.Ellison2010} e gráficos de linha para os resíduos.

\subsubsection{Distribuição Geográfica}

Foram construídos mapas temáticas que apresentam as de acordo com a distribuição geográfica de cada uma \cite{Borcard2011}. \color{blue}\textit{(O Borcard traz essa questão de mostrar as variáveis conforme a localização de cada registro, porém, não do mesmo modo que fizemos, acredito que entre como citação aqui (?).)}\color{black}

\subsubsection{Análise de Covariância}

Comparações entre curvas serão feitas por análise de covariância \cite{NicholasJ.Gotelli;AaronM.Ellison2010,Borcard2011, Field2012}.

\subsubsection{Análise de Classificação}

Para a realização de uma análise multivariada, foram desenvolvidas matrizes de distância, valendo-se do valor de similaridade de Jaccard \cite{VALENTIN2000,greenacre2014multivariate}, fundamentados nestas matrizes foram construídos dendogramas hierárquicos para determinar grupos de cidades similares conforme a composição de espécies \cite{VALENTIN2000,greenacre2014multivariate,Borcard2011}.

A posteriori, serão elaborados mapas coloridos conforme os grupos descritos acima e será empregado o teste de mantel para comparar as matrizes de distância \cite{Borcard2011,VALENTIN2000}.
