\section{Introdução}

Ecologia é o estudo da distribuição e abundância dos organismos \cite{Begon2007}. Determinar os padrões espaciais e temporais da biodiversidade tem implicações diretas sobre o manejo de recursos e serviços naturais \cite{groom2006principles,Roque2018}, consideradas as diferentes escalas em que ela se manifesta – desde o nível molecular até o nível da paisagem, incluindo a riqueza de espécies \cite{magurran2013measuring}. No período em que parte considerável da biota do planeta está ameaçada pela atividade humana, tal conhecimento pode reduzir efeitos que culminem em extinções prematuras \cite{groom2006principles}.

Porém, o número de pesquisadores aptos a conduzir a descrição pormenorizada dos referidos padrões é exígua ante a demanda \cite{Amano2016,Greenwood2007}. Por exemplo, embora haja uma diversidade notável de aves que, no estado de São Paulo, são listadas desde
o final do século XIX \cite{Silveira2011}, a ausência de uma quantidade abundante de registros limita o monitoramento e a conservação da avifauna \cite{Amano2016}. Portanto, tornam-se necessárias estratégias que otimizem tempo e esforço destinados a descrever padrões espaciais \cite{Amano2016,Greenwood2007}. O recrutamento massivo de leigos para o cumprimento de alguma das etapas desse tipo de levantamento é uma delas \cite{Lepczyk2005,Phillips2014,Tredick2017,Horns2018}. Reconhece-se esse esforço como ciência cidadã \cite{Kullenberg2016} que, além de proporcionar informações que auxiliam o estudo da Ecologia, promove a premência da conservação e educação ambiental \cite{DiasdaSilva2019,Bonney2016}. 

Esses cidadãos cientistas atuam principalmente na coleta de informações em campo, aprimorando gradualmente sua performance \cite{Phillips2014,Kieslinger2019}, sob tutela de um cientista profissional (com formação acadêmica e vinculado a algum órgão de pesquisa). A qualidade e validade dos bancos de dados assim obtidos deve ser testada para a plena aplicação dos resultados advindos de sua interpretação \cite{Phillips2014,Tredick2017,Kieslinger2019}. Devido à alta quantidade de indivíduos ativos envolvidos, estes bancos de dados podem reunir uma quantidade massiva de registros \cite{Alexandrino2018}.

Dadas as características marcantes e diagnósticas de grande parte das espécies de aves, sua determinação é plenamente possível por um iniciado com treino moderado. A disseminação global da atividade de birdwatching indica o apelo popular desse táxon \cite{Lepczyk2005,Alexandrino2018}. A ocorrência de populações de aves em determinada localidade é empregada amplamente como um indicador de condição ambiental \cite{Lepczyk2005,Greenwood2007,Schubert2019}, inclusive como um descritor alternativo e correlacionado à diversidade de outros grupos zoológicos e botânicos. Angariar a contribuição de indivíduos ordinários para o mapeamento seria, nesse caso, um modo de descrever o panorama do sistema ecológico usando essa mão-de-obra abundante e motivada  \cite{Lepczyk2005,Klemann-Junior2017,Alexandrino2018}.


Contudo, a ciência cidadã apresenta limitações. Para o monitoramento de aves, a primeira a se tratar é que os registros destas bases são dados de acordo com as coordenadas municipais e não são georreferenciados; entretanto, não é por esse motivo que devem ser desconsiderados \cite{Neto2017}. Outra limitação seria a incapacidade dos observadores de acessar terras privadas onde residem grande parte das espécies raras \cite{Lepczyk2005}. Embora a ciência cidadã tenha uma vasta capacidade de reunir dados, é preciso cautela quanto a sua utilização em projetos que exijam rigor científico \cite{Kieslinger2019}.

Portanto, faz-se necessária a avaliação de se (e quanto) essa descrição baseada no trabalho dos cientistas-cidadãos desvia-se daquela obtida sem sua participação, valendo-se apenas dos esforços dos cientistas profissionais \cite{Klemann-Junior2017}. Esse estudo tem como finalidade discutir essas questões, baseado em duas bases de dados do estado de São Paulo. O sítio SpeciesLink (SLI) reúne registros em coleções biológicas institucionais adquiridas primordialmente durante a atividade de pesquisadores. Esse tem sido o tipo de fonte fundamental para estudos envolvendo a distribuição da biodiversidade \cite{Horns2018}. O sítio WikiAves (WAV) reúne registros fotográficos e fonográficos de espécies em território brasileiro conduzidos por populares e com curadoria sob regência de especialistas (\cite{Cunha2014}; WikiAves 2020). Até o dia 25/04/2020, contava com 3.119.856 registros de 33.918 contribuintes para 1.890 espécies.
